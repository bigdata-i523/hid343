\documentclass[sigconf]{acmart}

\usepackage{graphicx}
\usepackage{hyperref}
\usepackage{todonotes}

\usepackage{endfloat}
\renewcommand{\efloatseparator}{\mbox{}} % no new page between figures

\usepackage{booktabs} % For formal tables

\settopmatter{printacmref=false} % Removes citation information below abstract
\renewcommand\footnotetextcopyrightpermission[1]{} % removes footnote with conference information in first column
\pagestyle{plain} % removes running headers

\newcommand{\TODO}[1]{\todo[inline]{#1}}

\begin{document}

\title{Big Data Applications and Autonomous Vehicles}

\author{Borga Edionse Usifo}
\affiliation{%
\institution{Indiana University}
\city{Bloomington} 
\state{Indiana} 
\postcode{47408}
}
\email{busifo@iu.edu}

\renewcommand{\shortauthors}{B. Usifo et al.}

\maketitle

\begin{abstract}
We will explain the importance of autonomous vehicles, Big Data applications used on these vehicles, and several computational methods used for achieving successful autonomy. 
\end{abstract}

\section{Introduction}
The way of life is changing every day with the help of technological advancements. We keep hearing more and more by companies how they are trying to make the computers to think and react like human beings. Autonomous vehicles are one of the products of these advancements. 
\par The main difference of learning between humans and computer is the way of gathering experience. Humans learn from experience but computers learn from the data. This is why data is the most fundamental aspect for the computer to do given task. We show about how computers process data what kind of analytic used for processing and learning from data and importance of big data.    

\section{Importance of Autonomous Vehicles}

Autonomous vehicles are essential, and it is the future of driving method to going A to B. There are several reasons for it which will change the future of driving. Before we go into detail about autonomous vehicles we need to learn types of autonomous vehicles which are listed below: 

\begin{description}
    \item[Level 0(no automation):] The driver responsible for all aspects of vehicle instruments while monitoring road con\-di\-ti\-ons\cite{hamzah}.
    \item[Level 1(function-specific automation):] At this level, automation needs to have at least one function to be automated, and functions need to be independent if there is more than one automated function\cite{hamzah}.
    \item[Level 2(combined-function automation):] This level re\-qu\-ir\-es a minimum of two automated functions to perform a task for the driver\cite{hamzah}.
    \item[Level 3(limited self-driving automation):] The driver, mu\-st have control of all the safety-related features under certain conditions while the vehicle is monitoring changes\cite{hamzah}.
    \item[Level 4(full self-driving automation):]At level 4 vehicle will monitor conditions of the environment and perform all the critical driving actions for the driver\cite{hamzah}.
\end{description}

\subsection{Safety Aspect}
Increasing safety features in motor vehicles is decreasing the number of crashes, if we look at the data from U.S. Census Bureau we can see that from 1980 to 2009 we have decreased in accidents. We can not tie all these decreases into technology because road structure and personal education also increased. According to the National Highway Traffic Safety Administration(NHTSA)
 
 \begin{figure}[!ht]
  \centering
      \includegraphics[width=\columnwidth]{images/picture5.png}
  \caption{Data from BTS(2013) includes all highway transportation modes: pass anger car, light truck, motorcycle, large truck, and bus. Fatalities include vehicle occupants for all highway modes, as well as pedestrians and cyclists}\label{fig:NHTSAaccidentreport}
\end{figure}

 \begin{figure}[!ht]
  \centering
      \includegraphics[width=\columnwidth]{images/picture6.png}
  \caption{Data from the Bureau of Transportation Statistics(BTS,2013) includes all highway transportation modes: passenger car, light truck, motorcycle, large truck, and bus. Crashes involving two or more motor vehicles are counted as one ``crash`` by the U.S. DOT, so total crashes shown  here are fewer than the sum of individual vehicles involved. Injuries include vehicle occupants for all highway modes as well as pedestrians and cyclists.}\label{fig:NHTSAaccidentreport2}
\end{figure}

\begin{itemize}

\item ``The economic cost of motor vehicle crashes that occurred in 2010 totaled 242 billion. This is equivalent to approximately 784 for every person living in the United States and 1.6 percent of the U.S Gross Domestic Product\cite{lawrance}.``

\item ``Some 3.9 million people were injured in 13.6 million cr\-as\-hes in 2010, including 32,999 deaths. Twenty-four percent of these injuries occurred in crashes that were not reported to police\cite{lawrance}.``

\end{itemize}


\subsection{Economic Aspect}
Beside from saving money from increasing the safety while decreasing the economic cost of motor vehicles, autonomous vehicles can improve many aspects of business and government supply chain industry, gas usage, time of commuting, and productivity.

\par According to RAND Corporation research, benefits of autonomous vehicles which includes productivity, gas consumption, increased safety aspects, improved mobility outperforms the disadvantages of autonomous vehicles\cite{RAND}.

\subsubsection{Fuel Consumption}As technology improves every day, we see live traffic events in our navigation apps, and optimally its steering individuals to go to different directions based on traffic events to eliminate any waste from commuting. With the help of autonomous vehicles, this live data directly go to intelligent vehicle systems and an autonomous car will steer their directions without the need of human interaction. An intelligent system like this improve the fuel consumption and decrease the commuting time from point A to B.

\begin{figure}[!ht]
  \centering
      \includegraphics[width=\columnwidth]{images/picture7.png}
  \caption{Range of fuel economy improvements for conventional, hybrid, and autonomous Cars}\label{F:fuelconsumption}
\end{figure}

\subsubsection{Supply Chain} As of year, 2017 majority of companies essential success to stay in competitive is the supply chain.  Autonomous vehicles will have a high impact on supply chain distributions because of ability to operate 24/7 in right circumstances, and faster travel times. Self-driving vehicles will also help the current driver shortage situation in supply chain businesses.

    ``In 2014 trucking industry was short 38,000 drivers. The shortage expected to reach nearly 48,000 in 2015 and if the current trend holds, shortage may balloon to almost 175,000 by 2024\cite{ATA}.``

\begin{figure}[!ht]
  \centering
      \includegraphics[width=\columnwidth]{images/picture8.png}
  \caption{Forecast of truck driver shortage}\label{F:small}
\end{figure}

\subsubsection{Productivity}Autonomous vehicle will also give people to do multitasking abilities for productivity improvements. Individuals will have more free time to do other tasks. 

``Currently, in the U.S., the average occupant of a light-duty vehicle spends about an hour a day traveling time that could potentially be put to more productive use. Indeed, increased productivity is one of the expected benefits of self-driving vehicles\cite{michigan}.``



\section{How it related to big data and what does Big mean?}
\par Data can come from all types of resources. In our case, it is sensors, signals, cameras, customer behaviors and many others resources. This data can be structured and unstructured dependent on where it is coming from while the term BIG refers to the volume of data it may also refer to techniques and tools that have been used to process this significant amount of data this tools can vary from cloud computing, visualization techniques to artificial intelligence procedures for analyzing\cite{www-webo}.

\par The essential success of autonomous vehicles depends on data. The more data they have the correct decision can the autonomous vehicles do. As we stayed before this data comes from the variety of places some of them are sensors, GPS signals, cameras, internet connectivity. All this data helps the car to make intelligent decisions while analyzing those data, without the data it will never successfully reach to the destination\cite{www-kdnuggets}.

\par Additionally, companies are using ``big data to optimize customer experience and operational safety ultimately laying the groundwork for the fully autonomous vehicles\cite{www-webo}.`` Companies can get collect data about customer driving habits by integrating additional sensors to its cars\cite{www-hb.org}. This connectivity to Big Data platform can give companies advantages over deploying new features to their cars. One another importance of Big Data connectivity for autonomous cars is the ability to transfer learning experience to other autonomous vehicles in other words when one autonomous vehicle learns from data and road conditions then that data can be transferable millions of other autonomous vehicles in contrast to individual experience which stays with the person \cite{www-hb.org}.

\begin{figure}[!ht]
  \centering
      \includegraphics[width=\columnwidth]{images/picture3.png}
  \caption{How big data and connectivity works}\label{F:connectivity}
\end{figure}

\begin{figure}[!ht]
  \centering
      \includegraphics[width=\columnwidth]{images/picture2.png}
  \caption{Process of connectivity}\label{fig:hbsconnectivitysteps}
\end{figure}


\section{Analytics Used on Autonomous Vehicles}
In this topic we will examine some of the methods that used in autonomous vehicles and these will include Machine Learning, Deep Learning, Artificial Intelligence.


\subsection{Machine Learning}

Machine learning widely used for many applications. Some of this applications include image and voice recognition, spam detection, fraud detection, the stock market, teaching a computer how to play chess, and, off course self-driving cars. 
\par Machine learning is teaching computers to learn to perform a task from past experiences this experience comes from data. Self-driving cars equipped with ECU (Electronic Control Units). These ECUs process data from sensors like Lidar, radars, cameras or the IoT(Internet of Things) and they are equipped with machine learning algorithms to make decisions in different conditions\cite{www-kdnuggets}. These decisions vary from adjusting the speed with different driving conditions to recognizing the pedestrian movement on the road. 

\begin{figure}[!ht]
  \centering
      \includegraphics[width=\columnwidth]{images/picture4.jpg}
  \caption{World from an eye of autonomous vehicle}\label{F:Lidar}
\end{figure}

\subsubsection{Should we store this data in someplace or analyze it simultaneously}
Current technology and Big Data methods allows self-driving or any other autonomous vehicles to analyze data on the go\cite{www-sas}.
    ``Analyze it on the fly. Rather than bringing data to the storage and analytics, bring the analytics to the data\cite{www-sas}.``

\subsection{Conventional Neural Networks}

Conventional Neural Networks (CNN)\cite{nvidia} used in pattern recognition applications. The significant advantage of CNN is that it can automatically learn features of the data from training examples\cite{nvidia}. This gives the significant advantage over learning features from image recognition. The image comes from camera system mounted on a car, after capturing images they will go through CNN, and after recognizing the features on the road, it will give the vehicle to steer itself based on computed steering command\cite{nvidia}. 

\begin{figure}[!ht]
  \centering
      \includegraphics[width=\columnwidth]{images/picture11.png}
  \caption{CNN process steps}\label{F:nvidiaCNNmodel}
\end{figure}

\subsection{Big Data for Predicting Safety Road Passage}
Big data can help to maximize safety aspects in self-driving cars by using Big Data mining and analytics. This kind of analytics will require vehicle and analytics to connect in cloud-based systems also this will require an entirely automated car, in this case, it is Level 4 which is a fully integrated self-driving car. This autonomy will give the vehicle to the ability to choose the safe passage at all times automatically as shown in Figure 10\cite{hamzah}.
\par The system still requires a driver to turn on the car and put the car id. After that, it requires the driver to put the destination. When Big Data engine receives all the required input, it will start predictions for road segments based on real-time Big Data analysis. If the cloud system does not predict any accidents in that road segments than vehicle continuous it is the destination as usual if the cloud system predicts any accidents then it reroutes the vehicle path to the destination\cite{hamzah}.
\par This kind of cloud system will also give the user to ability choose between fastest route or the best fuel consumption, but the safety is always going to be the first priority. (Below Fig) shows the pseudo-code for implementing in Big Data Engine.
\begin{figure}[!ht]
  \centering
      \includegraphics[width=\columnwidth]{images/picture9.png}
  \caption{Big data road passage safety process steps}\label{F:architecture}
\end{figure}

\begin{figure}[!ht]
  \centering
      \includegraphics[width=\columnwidth]{images/picture10.png}
  \caption{Pseudo-code for road passage safety}\label{F:pseudo-code}
\end{figure}

\section{Conclusion}
    We showed importance of autonomous vehicle and Big Data based applications on an autonomous vehicle is presented. Insights about advantages of autonomous vehicles had given. Several analytical approaches while using Big Data applications had shown.

\begin{acks}

The author would like to thank Dr. Gregor von Laszewski for his support and suggestions to write this paper.

\end{acks}

\bibliographystyle{ACM-Reference-Format}
\bibliography{report} 



\end{document}
